%%%%%%%%%%%%%%%%%%%%%%%%%%%%%%%%%%%%%%%%
%% MCM/ICM LaTeX Template %%
%% 2021 MCM/ICM           %%
%%%%%%%%%%%%%%%%%%%%%%%%%%%%%%%%%%%%%%%%
\documentclass[12pt]{article}
\usepackage{geometry}
\usepackage{amsmath}
\usepackage{chemarrow}
\usepackage{enumerate}
\usepackage{float}
\usepackage{graphicx}
\usepackage{subfigure}
\usepackage{listings}
\usepackage{xcolor}
\usepackage{ctex}
\usepackage{extarrows}
\usepackage{diagbox}
\usepackage{makecell}
\geometry{left=1in,right=0.75in,top=1in,bottom=1in}


%New colors defined below
\definecolor{codegreen}{rgb}{0,0.6,0}
\definecolor{codegray}{rgb}{0.5,0.5,0.5}
\definecolor{codepurple}{rgb}{0.58,0,0.82}
\definecolor{backcolour}{rgb}{0.95,0.95,0.92}

%Code listing style named "mystyle"
\lstdefinestyle{mystyle}{
  backgroundcolor=\color{backcolour},   commentstyle=\color{codegreen},
  keywordstyle=\color{magenta},
  numberstyle=\tiny\color{codegray},
  stringstyle=\color{codepurple},
  basicstyle=\ttfamily\footnotesize,
  breakatwhitespace=false,         
  breaklines=true,                 
  captionpos=b,                    
  keepspaces=true,                 
  numbers=left,                    
  numbersep=5pt,                  
  showspaces=false,                
  showstringspaces=false,
  showtabs=false,                  
  tabsize=2
}

%"mystyle" code listing set
\lstset{style=mystyle}
\usepackage{indentfirst}
\setlength{\parindent}{2em}
\bibliographystyle{unsrt}
%%%%%%%%%%%%%%%%%%%%%%%%%%%%%%%%%%%%%%%%
% Replace ABCDEF in the next line with your chosen problem
% and replace 1111111 with your Team Control Number
\newcommand{\Problem}{A}
\newcommand{\Team}{2016174}
%%%%%%%%%%%%%%%%%%%%%%%%%%%%%%%%%%%%%%%%

\usepackage{amsmath,amssymb,amsthm}

\newtheorem{theorem}{Theorem}
\newtheorem{corollary}[theorem]{Corollary}
\newtheorem{lemma}[theorem]{Lemma}
\newtheorem{definition}{Definition}

%%%%%%%%%%%%%%%%%%%%%%%%%%%%%%%%
\begin{document}
\graphicspath{{.}}  % Place your graphic files in the same directory as your main document
\DeclareGraphicsExtensions{.pdf, .jpg, .tif, .png}
\thispagestyle{empty}


%%%%%%%%%%% Begin Summary %%%%%%%%%%%
% Enter your summary here replacing the (red) text
% Replace the text from here ...

\vspace{20pt}
\centerline{{\Large \textbf{IGEM WP Model}}}
\vspace{15pt}

\centerline{{\large \textbf{Summary}}}
\vspace{7pt}



% to here
%%%%%%%%%%% End Summary %%%%%%%%%%%


%Begin your paper here

%====================目录页========================================
\newpage
\thispagestyle{empty}
\setcounter{page}{0}
%{\begin{center}\Large \textbf{Dynamic Model of ``Sweet Spot Effect''}\end{center}}
\tableofcontents                                                  %
\newpage                                                          %
%==================================================================
\section{Notations}
Here are some conventions about notation.
\begin{enumerate}
  \item words begin with $g$ represent gene(eg., gSos).
  \item words begin with $m$ represent mRNA(eg., mSos).
  \item words begin with $p$ represent promoter(eg., pSos).
  \item other words represent protein(eg., Sos).
  \item $\alpha$ represents transcription term(eg., $\alpha_{Sos}$).
  \item $\beta$ represents translation term(eg., $\beta_{Sos}$).
  \item $d$ represents decomposition term(eg., $d_{Sos}$).
\end{enumerate}

%%%%%%%%%%%%%%%%%%%%%%%%%%%%%%%%%% 2015 KU Model %%%%%%%%%%%%%%%%%%%%%%%%%%%%%%%%%%
\section{2015 KU Model}
Since it is a model reproduction work, I am not going to show a lot details about model design. Instead, I am going to point out the key to the model and the problems while I try to reproduce the result. Besides, I'd like to share what I gain in this process.
\subsection{Basic Ideas}
The basic model to describe the amount of chemical substances is \textbf{ODE Model}. If we totally figure out the transformation among chemical substances we concerned about, then we can use some submodels to describe their transformation rate, which is $\frac{\mathrm{d}amount}{\mathrm{d}t}$, the derivative of the amount.

In the following, I will show the basic chemical reactions and the submodels to describe them.
\subsection{Transcription and Hill Function}
According to central dogma, the first step of build a protein is transcription. If the transcription rate is a const, we can describe the process as $\frac{\mathrm{d}mRNA}{\mathrm{d}t} = r \cdot gRNA$ where $r$ is the transcription rate.

Actually, we always need a transcription factor. Only after the factor is binded to DNA, can they begin to transcribe. Hill Function is used to describe such a situation(cf., Fig. )
\begin{equation}
    \theta = \frac{[L]^n}{K_d^n + [L]^n} = \frac{1}{1 + (\frac{K_d}{[L]})^n} 
\end{equation}

$\theta$ is the amount of DNA bound by the protein, $[L]$ is the amount of protein, $K_d$ is the dissociation constant and $n$ is the Hill coefficient.

For the repressors, we are interested in how much of the DNA is still unbound and active: $\frac{1}{1 + (\frac{[Repressor]}{K_d})^n}$

For the activators, we are interested in how much of the DNA is bound and active: $ \frac{1}{1 + (\frac{K_d}{[Activator]})^n}$

So the ODE is modified as $$\frac{\mathrm{d}mRNA}{\mathrm{d}t} = r \cdot (\frac{1}{1 + (\frac{K_d}{[Activator]})^n} gRNA)$$
\subsection{Translation}
Translation is quite easy. We consider translate rate as a const, so the contribution of translation in derivative term is $$\frac{\mathrm{d}Protein}{\mathrm{d}t} = r_{Translation} \cdot mRNA$$
\subsection{Protein Association and Disassociation}
\subsection{Ping-Pong Bi-Bi}
\subsection{Spread}
\subsection{Reproduction Result and Problems}
\subsection{What have I gained}

%%%%%%%%%%%%%%%%%%%%%%%%%%%%%%%%%%% our model %%%%%%%%%%%%%%%%%%%%%%%%%%%%%%%%

\section{Our Model}
\subsection{Baisc View}
Our design is a feedback system in which the output protein $RhoA$ can detect the amount of input $TGF-\beta$ and change accordingly through a series of pathways.

The sketch of metabolic pathway is shown as following.

\subsection{Hill Function}
\subsection{Michaelis-Menten Equation}
\subsection{Protein Association and Disassociation}
\subsection{Results and Conclusion}
\subsection{Evaluation and Further Improvement}
%%%%%%%%%%%%%%%%%%%%%%%%%%%%%%
\end{document}
